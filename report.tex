\documentclass[a4paper, 11pt, nofonts, nocap, fancyhdr]{ctexart}

\usepackage[margin=60pt]{geometry}

\setCJKmainfont[BoldFont={方正黑体_GBK}, ItalicFont={方正楷体_GBK}]{方正书宋_GBK}
\setCJKsansfont{方正黑体_GBK}
\setCJKmonofont{方正仿宋_GBK}

\CTEXoptions[today=small]

\pagestyle{plain}

\usepackage{amsmath}
\usepackage{listings} 

\lstset{
  basicstyle=\ttfamily,
  columns=fullflexible,
  emph={interface, type, struct, uint8, bool, uint16, byte},
  emphstyle=\bfseries\ttfamily,
  keywordstyle=\bfseries
}

% \fancyhead[L]{\small{team name}}
% \fancyhead[C]{\small{FSTC 2014 - 05 - 训练报告}}
% \fancyhead[R]{\small{2014年8月2日}}

\title{算法设计Project:de Bruijn图上的编辑距离}
\author{陈镜融 \\ chenjr14@fudan.edu.cn}
\date{\today}

\begin{document}

\maketitle

\section{Abstract}

题目分为三个tasks,task1要求两个给定字符串的编辑距离;task2要求在k阶de Bruijn图上找到一条路径,其对应的字符串和给定字符串的编辑距离最小;task3是task2的数据范围放大版本。三个task都要输出具体方案。题目具体描述和数据在http://datamining-iip.fudan.edu.cn/ppts/algo/pj2017/

这次大作业,我完成了全部三个tasks,并尝试对task3进行了一定的优化。所有代码可以在https://github.com/crazyboycjr/algorithm-course-project上找到。

\section{Compile}

\begin{lstlisting}
g++ edit_distance1.cpp -o edit_distance1 -O2 -Wall
g++ edit_distance2.cpp -o edit_distance2 -O2 -Wall
g++ edit_distance3.cpp -o edit_distance3 -O3 -march=native
                                             -mtune=native -Wall -std=c++11 -g -mcmodel=large
\end{lstlisting}

其中\texttt{edit\_distance3.cpp}的编译可以增加\texttt{-fopenmp}选项,不过由于种种原因(估计是线程同步需要等待时间),使用openmp并没有能提高速度。可以考虑接下来将整个程序静态编译并加上\texttt{-pg},然后用\texttt{gprof}看一下分析结果。

\section{Run}

\begin{lstlisting}
ulimit -s unlimited
time ./edit_distance3 < task3.in > task3.out
\end{lstlisting}

注意:\texttt{./edit\_distance3}的运行需要较大的堆空间,实际运行大约60GB左右,整个程序运行时间在4h左右(常数优化前)。另外可以用\texttt{taskset}把进程绑到一个处理器上执行。

\section{Write-up}

\subsection{task 1}

经典的levenshtein distance问题,求一个字符串A经过增加一个字符,删除一个字符,替换一个字符变成字符串B的代价。每个操作代价为1。

用$f[i][j]$表示将字符串$A$的前缀$A[1..i]$变成字符串$B$的前缀$B[1..j]$,最少需要多少次操作。

为了方便说明,令$n = len(A)$, $m = len(B)$。$f[0][i]$和$f[i][0]$表示从空串变成另一个串的代价。

显然有$f[i][0] = i$; $f[0][j] = j$; 特殊的有$f[0][0] = 0$;

接下来为了计算$f[i][j]$,有最多四种可能的转移
\begin{equation*}
f[i][j] = min\left\{\begin{matrix}
f[i-1][j-1] & ,A[i] == B[j] \\ 
f[i-1][j] + 1 & \\ 
f[i][j-1] + 1 & \\ 
f[i-1][j-1] + 1 & 
\end{matrix}\right.
\end{equation*}

第一种转移表示能直接匹配,不需要额外代价,剩下三种转移分别表示删除$A[i]$,在$A$中增加字符$B[j]$,替换$A[i]$为$B[j]$。转移结束后$f[n][m]$就是最终答案。

为了输出方案,我们只需要记录转移的路径。另$opt[i][j]$ = NOP/DEL/ADD/SUB,分别对应四种转移情况,在实际发生转移时,更新$opt[i][j]$为对应的值。这里要注意初始状态$f[i][0] = i$的转移可以看成是由$f[i-1][0]$转移而来,因此对应DEL转移,$f[0][i]$对应ADD转移。

有了$opt[i][j]$的值,就可以通过$opt[n][m]$往前倒推,每次根据当前是由上次那种转移过来,可以知道$n$,$m$的变化。同时可以知道当前在字符串$A$上做的操作是什么。这部分可以递归完成。当$n+m == 0$时,说明转移倒推到最开始了。因为答案不会超过字符串长度,因此递归深度不会特别大。

时间复杂度$O(n^{2})$,空间复杂度$O(n^{2})$,其中$n$为字符串最大长度。这个算法的空间可以继续优化,我们往下看


\subsection{task 2}


这个任务和task1任务不同。task2给出了一张图,图上每个节点是一个长度为$k(k \leq 30)$的字符串,两个节点之间存在有向边,当且仅当第一个字符串的长度为$k-1$的后缀,和第二个字符串长度为$k-1$的前缀完全相同。可以想象,图上一条长度为$l$的路径,构成了一个长度为$k+l-1$的字符串。现在给出字符串$A$,和$m$个长度为$k$的字符串,希望在图中找到一条路径,使得路径所对应的字符串和A的编辑距离最小。

首先说建图,这个过程没有什么难度,可以简单考虑用两个\texttt{unordered\_map<string, vector<int>>[2]}表示有哪些编号的字符串以key为前缀或者后缀,具体来说\texttt{M[0][prefix]}表示以字符串prefix为前缀的字符串编号列表,\texttt{M[1][suffix]}表示以suffix为后缀的字符串编号列表。由于题目中约束字符集大小只有4,在实际数据中没有重复字符串出现的情况下,这个列表长度不会超过4。所以存图也可以用前向星,邻接表之类的方法存(这里用什么方法存图会在后面常数优化章节进行比较)。

在不考虑空间的情况下,这个题目仍然可以考虑用动态规划,因为可以发现在图上走一步,其实只增加了一个字符,这与第一问中字符串上走一步,本质是没有什么不同的。于是我们修改转移状态,用$f[i][j][k]$表示在字符串$A$的前缀$A[1..i]$,变化成一个在图上走了$k$步的字符串,这个字符串最后停在$j$这个节点上,最少需要多少代价。

考虑初始状态和转移,我们发现在这个dp过程中,转移每次增加一个字符,而初始状态是上来一个字符串作为开头,没法归类到转移当中。于是初始状态需要提出来单独计算。

对于初始状态,实际上我们需要计算$f[i][j][1]$,其中$i$和$j$是变值,即对每个字符串$j$,计算任意长度的前缀和变换到字符串$j$的代价。
这一部分我们可以调用task1中实现的函数,不做任何优化考虑,时间复杂度为$O(nmk)$,其中$n$为字符串$A$的长度,$m$为图中节点数,$k$为每个节点上字符串的长度。

接下来仍然考虑四种转移,为了方便起见,用$tlen$表示题目描述中的$k$,即节点上每个字符串的长度,用$ns[j]$表示第$j$个字符串
\begin{equation*}
f[i][j][k] = min\left\{\begin{matrix}
f[i-1][y][k-1] & ,A[i] == ns[j][tlen] \\ 
f[i-1][j][k-1] + 1 & \\ 
f[i][y][k-1] + 1 & \\ 
f[i-1][y][k-1] + 1 & 
\end{matrix}\right.
\end{equation*}

$y$表示字符串$j$的前一个可能的节点,转移的时候取较小值转移。类似的,第一种对应直接能匹配,第二种对应删除字符$A[i]$,第三种对应在A中增加字符$ns[j][tlen]$,第四种对应替换。

在这种情况下,答案等于这个数组中的最小值。但实际上我们并开不下空间存储这样的状态,于是考虑修改一下状态的含义。$f[i][j][k]$即路径不超过$k$的所有停在$j$上的字符串,变化到$A$的最小代价,相当于对前面的状态做一个继承。于是多一种转移$f[i][j][k] = f[i][j][k-1]$即可。这样我们可以发现$f[i][j][k]$一定是由$f[x][y][k-1]$转移而来,于是第三维可以滚动。

但实际上,仅仅是上面这样的考量是不够的。考虑这样一种情况,有三个字符串分别为"abb","bbc","bbb",字符串2在和$A$的某个前缀比较时,可能从字符串1转移过来,也可能从字符串3转移过来,而不恰当的转移顺序会丢掉1先增加一个字符转移到3,再由3增加一个字符到2的情况。作为不,考虑转移顺序的一个workaround,考虑图中可能出现环的情况,对$f[i][x][]$最少需要重复转移$tlen$次,才能保证dp的正确性。

对于方案的输出,则需要多记录上一次是从哪个节点转移过来的,一样可以递归输出。

因此该算法的时间复杂度为$O(nmk)$,空间复杂度为$O(nm)$, 对于task2的数据,可以在十几秒级别的时间出解。

\subsection{task 3}

面对task3的数据规模,我们主要解决task2算法时间和空间上的不足。

先考虑空间上的优化
\begin{enumerate}
\item task1当中的dp数组是可以滚动的,所以这里只需要$O(m)$的空间
\item task2中的dp数组,第一维也是可以滚动的,于是这里的空间也只有$O(m)$。
\item 对于方案的输出,由于一共有$n*m$的状态,所以$n*m$的空间必不可少,对于$n=100000$, $m=1000000$来说,$n*m$个int需要大约400GB的内存空间,实际上,每种转移只有4种情况,对应2bit,而转移要记录上一个结点,又因为字符集大小只有4,且没有重复字符串,所以只关心上一个字符即可,也需要2bit,所以对于转移方案的记录,一个需要4bit,相比之前用int来存,我们节省了8倍的空间,于是这里空间大约需要50GB,在一台小型的服务器上已经可以承载。
\item 另外还有dp初始化需要记录结果,一共有$n*m$个结果需要保留,而实际上,由于数据随机,可以发现在$ns[i]$已经在$A$串中完全出现之后,其后面的值不需要再次计算。这里我们用$m$个vector,$bound[i][j]$表示第$i$个字符串,和$A$串的前缀$A[1..j]$的编辑距离,对于$bound[i][j]+tlen==j$之后的情况,可以不用存储。对于随机数据来说,在长度为100多的字符串中几乎就出现了$ns[i]$。所以这不但节省了空间,而且节省了dp初始化的计算时间。初始化时间和空间都降低为了$O(4mk^{2})$。
\end{enumerate}

在空间优化完成后,程序就已经可以跑了,经过估计大约需要60h可以出解。然而时间上任然可以继续优化。
\begin{enumerate}
\item 观察task2中$f[i][j][k]$,已经对k这一维的转移结果进行继承了。为何不干脆把这一维直接优化掉?实践发现这样是可行的。于是记录状态的数组变成了$f[i][j]$
\item 实际上当一个状态$f[i][j]$在某次转移后值已经不变了,那么由它去再次转移后面的值没有意义。所以我们可以考虑用两个队列,每个队列里存放待转移的节点。第一个队列我们依次计算转移,当一个节点转移成功后,将其后继状态,也就是这个结点可以去更新的状态,加入到下一个队列中,等待下一轮转移。经过这个优化,结果发现从原来每个结点需要迭代$tlen$次,降低到了平均迭代2次左右。在这个优化加上之后,跑出结果大约只需要4h。
\end{enumerate}

时间复杂度$O(4mk^{2}+4Cnm)$,空间复杂度$O(nm)$。这里的$C$最坏能达到$k$,但实际情况下$C$只有2左右。

\subsection{constant optimization}

为了进一步优化,我们可以考虑从常数方面,和利于多线程并行方面进行。
\begin{enumerate}
\item 存图方面,因为需要大量次数的枚举图上相邻边,可以考虑用连续空间的存储代替前向星存储。
\item 考虑cpu cache,dp中的滚动数组应当将小的一维作为第二维。这样可以有效加快计算的速度。实测光是初始化阶段就快了几十倍。
\item 考虑用openmp来优化,这部分我做了一些工作,但效果不理想,我觉得主要有三个原因。
\begin{enumerate}
	\item 第一个原因是每个线程做的任务太少,反而线程之间来回取任务占用了更多时间,这个已经通过增大每个线程执行的工作量来解决了,但也只能是一个模糊的数值
	\item 第二个是线程之间同步,比如分出12个线程执行一个for循环,但for循环结束之后,需要等待所有线程都执行完,这里应该会花费不少时间
	\item 第三个是for循环虽然并行了,但访问内存速度是瓶颈,由于数据巨大,多线程得写内存范围伤害了cache的局部性,导致速率得不到提高。
\end{enumerate}
\end{enumerate}

\section{Others}

感谢姜峻岩同学提供的数据,对我调试和测试提供了极大的帮助。

\end{document}
